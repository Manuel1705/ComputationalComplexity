\section{Database queries}
\subsection{Complexity measures}
There are different complexity measures for the queries:
\textbf{Combined Complexity} where both the queries and the database are part of the input; \textbf{Data complexity} where only the database is part of the input (we consider the size of the query negligible); \textbf{Expression complexity} where only the query is part of the input (given a database we want to analyze what happens what happens as the query complexity increases)
\begin{defbox}[Theorem]
    \begin{itemize}
        \item Combined complexity: \textbf{PSPACE}-complete
        \item Expression complexity: \textbf{PSPACE}-complete
        \item Data complexity: \textbf{L}-complete $(\textbf{L}\subseteq \textbf{P})$
    \end{itemize}
\end{defbox}
The \textbf{congiuntive queries} are queries without unions and negations. When we only use this type of queries, the complexity is: 
\begin{itemize}
        \item Combined complexity: \textbf{NP}-complete
        \item Expression complexity: \textbf{NP}-complete
        \item Data complexity: $\textbf{AC}_0-complete(\textbf{AC}_0\subseteq\textbf{L}\subseteq \textbf{P})$
\end{itemize}

